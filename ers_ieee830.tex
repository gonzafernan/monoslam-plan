% Created 2020-04-24 vie 23:03
% Intended LaTeX compiler: pdflatex
\documentclass[12pt,a4paper, twosite]{article}
\usepackage[utf8]{inputenc}
\usepackage[T1]{fontenc}
\usepackage{graphicx}
\usepackage{grffile}
\usepackage{longtable}
\usepackage{wrapfig}
\usepackage{rotating}
\usepackage[normalem]{ulem}
\usepackage{amsmath}
\usepackage{textcomp}
\usepackage{amssymb}
\usepackage{capt-of}
\usepackage{hyperref}
\usepackage[left=2.00cm, right=2.50cm, top=2.50cm, bottom=2.00cm]{geometry}
\usepackage{fancyhdr}
\fancyhead[RO,LE]{\thepage}
\fancyhead[LO]{\emph{\uppercase{\leftmark}}}
\fancyfoot{}
\renewcommand{\headrulewidth}{1.0pt}
\pagestyle{fancy}
\date{}
\title{ERS: Sistema de SLAM Visual/Inercial monocular}
\hypersetup{
 pdfauthor={},
 pdftitle={IEEE-830},
 pdfkeywords={},
 pdfsubject={},
 pdfcreator={Emacs 26.2 (Org mode 9.1.9)}, 
 pdflang={English}}
\begin{document}

\maketitle
\tableofcontents

\newpage

\section{Introducción}
\label{sec:org60390fa}

% En esta sección se proporcionará una introducción a todo el
% documento de Especificación de Requisitos Software (ERS). Consta de
% varias subsecciones: propósito, ámbito del sistema, definiciones,
% referencias y visión general del documento.


\subsection{Propósito}
\label{sec:org434c3ef}

% En esta subsección se definirá el propósito del documento ERS y se
% especificará a quien va dirigido.

Este documento consiste en la especificación de requerimientos de software para un
\textit{Sistema de Localización y Mapeo Simultáneo Visual/Inercial}.
El documento está dirigido a los desarrolladores a cargo del análisis, diseño,
implementación y evaluación (\textit{testing}).

\subsection{Ámbito del sistema}
\label{sec:org12e44a1}

% En esta subseccion:

% \begin{itemize}
% \item Se podrá dar un nombre al futuro sistema (p.ej. MiSistema)

% \item Se explicará lo que el sistema hará y lo que no hará.

% \item Se describirán los beneficios, objetivos y metas que se espera
% alcanzar con el futuro sistema.

% \item Se referenciarán todos aquellos documentos de nivel superior (p.e en
% Industria de sistemas, que incluyen hardware y software, deberían
% mantenerse la consistencia con el documento de especificaciones de
% requisitos globales del sistema, si existe)
% \end{itemize}

El nombre asignado al sistema es "\textit{Polyphemos}", en referencia al cíclope del poema
griego la \textit{Odisea}. 

El sistema consistirá en un módulo de software con la capacidad de obtener información de una única
cámara y una unidad de medición inercial (IMU por sus siglas en inglés), ejecutar un algoritmo de
localización y mapeo simultáneo (SLAM por sus siglas en inglés), para luego brindar a la aplicación
del usuario información sobre su ubicación y las características espaciales del entorno en el que se
encuentra.

El usuario podrá integrar el módulo en su aplicación específica y utilizarlo, por ejemplo, para
realimentar un lazo de control de un robot móvil y dotarlo así de autonomía para la navegación en su
espacio de trabajo. El sistema no incluye ningun tipo de algoritmo de control o capacidad para
operar los actuadores de un sistema en particular.

El sistema funcionará sobre un microcontrolador ARM Cortex-M4, y se comunicará con un sistema en la
PC del usuario que utiliza el framework de desarrollo de sistemas robóticos distribuidos ROS 2.
El proyecto \textit{Polyphenos} ofrece a su vez un sistema con dichas características para la
evaluación, calibración y aprendizaje del usuario sobre el funcionamiento del módulo de software.

El proyecto \textit{Polyphemos} será \textit{open-source} por lo que tendrá la documentación
y herramientas adecuadas para que la comunidad pueda ser usuaria y colaboradora del proyecto en una
etapa estable del éste.

\subsection{Definiciones, Acrónimos y Abreviaturas}
\label{sec:orgb158e36}

% En esta subsección se definirán todos los términos, acrónimos y
% abreviaturas utilizadas en la ERS.

\begin{enumerate}
  \item Definiciones:
  \begin{itemize}
    \item Plataforma móvil: Sistema físico sobre el que se encuentra el hardware del proyecto
    (los sensores y el microcontrolador) que será desplazado sobre el espacio de trabajo que posea
    la aplicación del usuario.
    \item Sistema usuario: 
  \end{itemize}
  \item Acrónimos:
  \begin{itemize}
    \item SLAM: \textit{Simultaneous Localization and Mapping}, Localización y Mapeo Simulatáneo.
    \item IMU: \textit{Inertial Measurement Unit}, Unidad de Medición Inercial.
    \item ROS: \textit{Robot Operating System}
  \end{itemize}
  \item Abreviatura:
  \begin{itemize}
    \item Std: estándar
  \end{itemize}
\end{enumerate}


\subsection{Referencias}
\label{sec:org62711e0}

En esta subsección se mostrará una lista completa de todos los
documentos referenciados en la ERS


\subsection{Visión general del documento}
\label{sec:orgdaca22c}

% Esta subsección describe brevemente los contenidos y la organización
% del resto de la ERS.

El documento se elaboró según lo especificado en el estándar IEEE Std. 830-1998.

En la sección \ref{sec:orgc1c4017} se describe la perspectiva del producto y el contexto
en el que se encuentra bajo un proyecto compuesto por diferentes módulos que interaccionan con
el descripto en el documento, se enumeran las funciones principales que cumplirá y en qué tipos
de usuarios se enfoca el proyecto y cuáles son sus características. Además, se describen las
restricciones que posee el desarrollo del producto y cuáles son los supuestos en la
planificación de este. Al final de la sección se enumeran requisitos que se tendrá a futuro y
que deben considerarse en el desarrollo actual.

En la sección \ref{sec:org40573d1} se enumera en detalle todos los requerimientos de software
del producto en análisis.

\section{Descripción general del documento}
\label{sec:orgc1c4017}

% En esta sección se describen todos aquellos factores que afectan al
% producto y a sus requisitos. No se describen los requisitos, sino su
% contexto. Esto permitirá definir con detalle los requisitos en la
% sección 3, haciendo que sean más fáciles de entender.

% Normalmente, esta sección consta de las siguientes subsecciones:
% Perspectiva del producto, funciones del producto, características de
% los usuarios, restricciones, factores que se asumen y futuros
% requisitos.


\subsection{Perspectiva del producto}
\label{sec:org24980a8}

% Esta subsección debe relacionar el futuro sistema (producto
% software) con otros productos. Si el producto es totalmente
% independiente de otros productos, también debe especificarse
% aquí. Si la ERS define un producto que es parte de un sistema mayor,
% esta subsección relacionará los requisitos del sistema mayor con la
% funcionalidad del producto mayor y el producto aquí descripto. Se
% recomienda utilizar diagramas de bloques.

El producto de software aquí especificado estará vinculado a un prototipo de hardware
ofrecido por el mismo proyecto, compuesto por la plataforma embebida, es decir,
un microcontrolador Cortex-M, una cámara OV9655 y la IMU integrada por un MPU6050 (acelerómetro
y giróscopo) y un magnetómetro HMC5883L (GY273).

El módulo de software será diseñado de forma modular para permitir que los usuarios lo
adapten a su hardware específico en caso de que este difiera del utilizado en el proyecto, es decir, el usuario será capaz de reemplazar los drivers por defecto con los correspondientes a
los sensores que posea.

El producto interactuará con el sistema usuario, que consiste en un sistema diseñado con el
framework ROS 2. Dicho sistema lo provee el usuario o el proyecto, y se ejecuta en la PC del
usuario bajo un sistema operativo de propósito general (por ejemplo Linux). El sistema posee
herramientas de visualización y un estándar que respetará para comunicarse con el módulo de 
software en el sistema embebido.

La comunicación con el sistema usuario es bidireccional: el sistema usuario recibirá la información
procesada por el producto y también podrá ejecutar rutinas de calibración y configuración del
módulo de software.

El sistema usuario brindado por el proyecto tiene como objetivo proveer al usuario herramientas de
evaluación, calibración y aprendizaje del uso del producto. También permite al usuario tomarlo
como base para el desarrollo de su propio sistema usuario.

\subsection{Funciones del producto}
\label{sec:orgaf51da6}

% En esta subsección de la ERS se mostrará un resumen, a grandes
% rasgos, de las funciones del futuro sistema, por ejemplo, en una ERS
% para un programa de contabilidad, esta subsección mostrará que el
% sistema soportará el mantenimiento de cuentas, mostrará el estado de
% las cuentas y facilitará la facturación, sin mencionar el enorme
% detalle que cada una de estas funciones requiere.

% Las funciones deberán mostrarse de forma organizada, y pueden
% utilizarse gráficos, siempre y cuando dichos gráficos reflejen las
% relaciones entre funciones y no el diseño del sistema.

\begin{enumerate}
  \item Permitirá la configuración de la IMU con consignas provenientes del sistema usuario.
  \item Proveerá al sistema usuario los datos obtenidos con la IMU sin procesar.
  \item Proveerá los datos procesados de posición y orientación del sistema físico al sistema
  usuario.
  \item Permitirá la calibración de la IMU a través de rutinas determinadas. 
  \item Proveerá los datos obtenidos por la cámara al sistema usuario.
  \item Generará un mapa del entorno del sistema desde cero o sobre información previa y brindará
  al sistema usuario el correspondiente mapa en conjunto con la ubicación del sistema en el mismo.
\end{enumerate}

\subsection{Características de los usuarios}
\label{sec:orga40b0ee}

% Esta subsección describirá las características generales de los
% usuarios del producto, incluyendo nivel educacional, experiencia y
% experiencia técnica.

Existen diferentes perfiles de usuario final del producto. Se pueden destacar los siguientes:

\begin{itemize}
  \item Usuario con conocimiento técnico en la aplicación donde incorporará el sistema SLAM, por
  ejemplo la robótica móvil. Po see conocimientos básicos sobre el framework ROS 2 y el uso del
  sistema operativo de propósito general donde se ejecutará el sistema usuario. No tienen
  necesariamente conocimiento técnico de electrónica y sistemas embebidos. Su principal intención
  es incorporar el sistema SLAM como caja negra en su propio producto.
  \item Usuario con conocimiento sobre electrónica y sistemas embebidos, con intención de utilizar
  el sistema SLAM con los sensores de su elección. Utiliza el módulo de software como caja negra e
  integra sus propios drivers en el sistema. Tiene conocimiento básico del sistema operativo de
  propósito general donde se ejecutará el sistema usuario, pero no posee conocimiento sobre el
  framework ROS 2.
  \item Usuario con conocimiento sobre las tácnicas de SLAM. Su intención es implementar algoritmos
  propios de SLAM y utilizar el producto como plataforma de desarrollo. Es el único usuario que
  utiliza el producto como caja blanca (lo que es posible por ser un proyecto \textit{open
  source}). No tiene conocimiento técnico en electrónica o sistemas embebidos, su principal interes
  es el hardware e interfaces de software ya definidas.
\end{itemize}

\subsection{Restricciones}
\label{sec:org5ca5790}

% Esta subsección describirá aquellas limitaciones que se imponen
% sobre los desarrolladores del producto.

% \begin{itemize}
% \item Políticas de la empresa.

% \item Limitaciones del hardware.

% \item Interfaces con otras aplicaciones.

% \item Operaciones paralelas.

% \item Funciones de auditoría

% \item Funciones de control.

% \item Lenguaje(s) de programación

% \item Protocolos de comunicación.

% \item Requisitos de habilidad.

% \item Criticalidad de la aplicación.

% \item Consideraciones acerca de la seguridad.
% \end{itemize}

\begin{itemize}
  \item El software del proyecto debe desarrollarse en lenguaje de programación C.
  \item La compilación del software del proyecto debe realizarse con el \textit{toolchain}
  de GCC para la arquitectura.
  \item El sistema debe utilizar herramientas de evaluación del software desarrollado
  (\textit{testing}).
  \item Los desarrolladores deben tener experiencia en el uso del sistema de control de
  versiones Git.
  \item Se debe utilizar Github como servidor de repositorio Git remoto.
\end{itemize}

\subsection{Suposiciones y dependencias}
\label{sec:org0ae23fe}

% Esta subsección de la ERS describirá aquellos factores que, si
% cambian, pueden afectar a los requisitos. Por ejemplo, los
% requisitos pueden presuponer una cierta organización de ciertas
% unidades de la empresa, o pueden presuponer que  el sistema correrá
% sobre cierto sistema operativo. Si cambian dichos detalles en la
% organización de la empresa, o si cambian ciertos detalles técnicos,
% como el sistema operativo, puede ser necesario revisar y cambiar los
% requisitos. 

Para el desarrollo del producto se supone que:

\begin{itemize}
  \item Los sensores seleccionados tienen un desempeño que permite como mínimo establecer sobre
  el primer prototipo de hardware que el software desarrollado es funcional.
  \item El microcontrolador seleccionado tiene la capacidad de cómputo suficiente para ejecutar
  una primera versión del algoritmo de SLAM.
\end{itemize}


\subsection{Requisitos futuros}
\label{sec:org33cfcdb}

% Esta subsección esbozará futuras mejoras al sistema, que podrán
% anlizarse e implementarse en el futuro.

\begin{itemize}
  \item En el futuro se incorporará la posibilidad de añadir una FPGA al hardware para optimizar el
  algoritmo utilizado por el sistema y mejorar su desempeño.
  \item En el futuro se desea incorporar otras arquitecturas de microprocesadores además de
  ARM Cortex-M4.
\end{itemize}

\section{Requisitos específicos}
\label{sec:org40573d1}

% Esta sección contiene los requisitos a un nivel de detalle suficiente
% como para permitir a los diseñadores diseñar un sistema que
% satisfaga estos requisitos, y demuestren si el sistema satisface, o
% no, los requisitos. Todo requisito aquí especificado describirá
% comportamientos externos del sistema, perceptibles por parte de los
% usuarios, operadores y otros sistemas. Esta es la sección más larga
% e importante de la ERS. Deberán aplicarse los siguientes principios:

% \begin{itemize}
% \item El documento debería ser perfectamente legible por personas de muy
% distintas formaciones e intereses.

% \item Deberán referenciarse aquellos documentos relevantes que poseen
% alguna influencia sobre los requisitos.

% \item Todo requisito deberá ser unívocamente identificable mediante algún
% código o sistema de numeración adecuado.

% \item Lo ideal, aunque en la práctica no siempre realizable, es que los
% requisitos posean las siguientes características: 

% \begin{itemize}
% \item \textbf{Corrección:} La ERS es correcta si y sólo si todo requisito que
% figura aquí(y que será implementado en el sistema) refleja alguna
% necesidad real. La corrección de la ERS implica que el sistema
% implementado será el deseado.

% \item \textbf{No ambiguos:} Cada requisito tiene una sola interpretación. Para
% eliminar la ambigüedad inherente a los requisitos expresados en
% lenguaje natural, se deberán utilizar gráficos o notaciones
% formales. En el caso de utilizar términos que, habitualmente,
% poseen más de una interpretación, se definirán con precisión en
% glosario.

% \item \textbf{Completos:} Todos los requisitos relevantes han sido incluidos en
% la ERS. Conviene incluir todas las posibles respuestas del sistema
% a los datos de entrada, tanto validos como no válidos.

% \item \textbf{Consistentes:} Los requisitos no pueden ser contradictorios. Un
% conjunto de requisitos contradictorios no es implementable.

% \item \textbf{Clasificados:} Normalmente, no todos los requisitos son igual de
% importantes. Los requisitos pueden clasificarse por importancia
% (esenciales, condicionales u opcionales) o por estabilidad (cambios
% que se espera que afecten al requisito). Esto sirve, ante todo,
% para no emplear excesivos recursos en implementar requisitos no
% esenciales.

% \item \textbf{Verificables:} La ERS es verificalble si y sólo si todos sus
% requisitos son verificables. Un requisito es verificable
% (testeable) si existe un proceso finito y no costoso para
% demostrar que el sistema cumple con el requisito. Un requisito
% ambiguo no es, en general, verificable. Expresiones como a veces,
% bien, adecuado, etc introducen ambigüedad en los
% requisitos. Requisitos como "en caso de accidente la nube tóxica
% no se extenderá más allá de 25km" no es verificable por el alto
% costo que conlleva.

% \item \textbf{Modificables:} La ERS es modificable si y sólo si se encuentra
% estructurada de forma que los cambios a los requisitos puedan
% realizarse de forma fácil, completa y consistente. La utilización
% de herramientas automáticas de gestión de requisito (por ejemplo
% RequisitePro o Doors) facilitan enormemente esta tarea.

% \item \textbf{Trazables:} La ERS es trazable si se conoce el origen de cada
% requisito y facilita la referencia de cada requisito a los
% componentes y de la implementación. La trazabilidad hacia atrás
% indica el origen (documento, persona, etc) de cada requisito. La
% trazabilidad hacia delante de un requisito R indica qué
% componentes del sistema son los que realizan el registro R.
% \end{itemize}
% \end{itemize}


\subsection{Interfaces externas}
\label{sec:orgfd5391f}

% Se describirán los requisitos que afecten a la interfaz de usuario,
% interfaz con otros sistemas (hardware y software) e interfaces de comunicaciones.

% PYPH

\begin{itemize}
  \item \textbf{[PYPH-RS-001]:} El sistema SLAM ofrecerá conexión con el hardware del sistema
  usuario a través de un bus serial con baud-rate 115200, 8 bits, 1 bit de stop, sin
  paridad y sin control de flujo.
  \item \textbf{[PYPH-RS-002]:} El sistema SLAM ofrecerá conexión inalámbrica con el hardware
  del sistema usuario a través de Bluetooth.
  \item \textbf{[PYPH-RS-003]:} El sistema SLAM establecerá un canal de comunicación compatible
  con el framework ROS 2 dedicado a la configuración del módulo y la información de debug.
  \item \textbf{[PYPH-RS-004]:} El sistema SLAM establecerá un canal de comunicación compatible
  con el framework ROS 2 asociado a la IMU que distinga la información vinculada a cada sensor
  de esta.
  \item \textbf{[PYPH-RS-005]:} El sistema SLAM establecerá un canal de comunicación compatible
  con el framework ROS 2 asociado a la cámara.
  \item \textbf{[PYPH-RS-006]:} El sistema SLAM establecerá un canal de comunicación compatible
  con el framework ROS 2 asociado a la posición y orientación de la plataforma móvil.
  \item \textbf{[PYPH-RS-007]:} El sistema SLAM brindará un servicio compatible con el framework
  ROS 2 que permita actualizar el mapa del entorno en el sistema usuario.
  \item \textbf{[PYPH-RS-008]:} El sistema SLAM definirá la interfaz que debe respetar el driver
  de la IMU y además se ofrecerá una implementación para la IMU MPU6050 en conjunto con el
  magnetómetro HMC5883L (GY273).
  \item \textbf{[PYPH-RS-009]:} El sistema SLAM definirá la interfaz que debe respetar el driver
  de la cámara y además se ofrecerá una implementación para el modelo OV9655.
\end{itemize}

\subsection{Funciones}
\label{sec:org307bb59}

% Esta subsección (quizás la más larga del documento) deberá
% especificar todas aquellas acciones (funciones) que deberá llevar a
% cabo el software. Normalmente (aunque no siempre) son aquellas
% acciones expresables como "el sistema deberá \ldots{}" Si se considera
% necesario, podrán utilizarse notaciones gráficas y tablas, pero
% siempre supeditadas al lenguaje natural, y no al revés.

% Es importante tener en cuenta que, en 1983, el estándar de IEEE 830
% establecía que las funciones deberían expresarse como una jerarquía
% funcional (en paralelo con los DFDs propuestas por el análisis
% estructurado). Pero el estándar de IEEE 830, en sus últimas
% versiones, ya permite organizar esta subsección de múltiples formas,
% y sugiere, entre otras, las siguientes:

% \begin{itemize}
% \item Por tipos de usuarios: 
%     Distintos usuarios poseen distintos requisitos. Para cada clase de
% usuario que exista en la organización, se especificarán los
% requisitos funcionales que le afecten o tengan mayor relación con
% sus tareas.
% \end{itemize}


% \begin{itemize}
% \item Por objetos:
%    Los objetos son identidades del mundo real que serán reflejadas en
% el sistema. Para cada objeto, se detallarán sus atributos y sus
% funciones. Los objetos pueden agruparse en clases. Esta organización
% de la ERS no quiere decir que el diseño del sistema siga el
% paradigma de Orientación a Objetos.
% \end{itemize}


% \begin{itemize}
% \item Por estímulos: 
%   Se especificarán los posibles estímulos que recibe el sistema y las
% funciones relacionadas con dicho estímulo.
% \end{itemize}


% \begin{itemize}
% \item Por jerarquía funcional: 
%    Si ninguna de las anteriores alternativas resulta de ayuda, la
% funcionalidad del sistema se especificará como una jerarquía de
% funciones que comparten entradas, salidas o datos internos. Se
% detallarán las funciones (entrada, proceso, salida) y las
% subfunciones del sistema. Esto no implica que el diseño del sistema
% deba realizarse según el paradigma de diseño estructurado.
% \end{itemize}


% Para organizar esta subsección de la ERS se elegirá alguna de las
% anteriores alternativas, o incluso alguna otra que se considere más
% conveniente. Deberá, eso sí, justificarse el porqué de tal elección.

Referidos a la IMU:

\begin{itemize}
  \item \textbf{[PYPH-RS-010]:} El sistema SLAM debe permitir mediante un mensaje en el canal de
  comunicación asociado a la IMU, configurar el rango de cada uno de los sensores que la
  componen, respetando las opciones disponibles de rango en cada uno de ellos.
  \item \textbf{[PYPH-RS-011]:} El sistema SLAM debe permitir mediante un mensaje en el canal de
  comunicación asociado a la IMU, seleccionar la representación numérica que el sistema debe
  utilzar al publicar los datos sin procesar con la IMU. La selección podrá darse entre un
  entero sin signo de 16 bits y un valor flotante de precisión simple.
  \item \textbf{[PYPH-RS-012]:} El sistema SLAM debe permitir mediante un mensaje en el canal de
  comunicación asociado a la IMU, configurar la frecuencia de muestreo en cada uno de los
  sensores de la IMU.
  \item \textbf{[PYPH-RS-013]:} El sistema SLAM debe permitir mediante un mensaje en el canal de
  comunicación asociado a la IMU, reiniciar el driver de la IMU.
  \item \textbf{[PYPH-RS-014]:} El sistema SLAM debe permitir mediante un mensaje en el canal de
  comunicación asociado a la IMU, realizar un chequeo de sanidad para comprobar que la IMU se
  encuentra disponible. Ante tal mensaje el sistema responderá con otro mensaje en el mismo
  canal de comunicación, con un identificador de error si no se encuentra disponible, o caso
  contrario con información del módulo de la IMU (por ejemplo, ID y configuración actual) para
  que el usuario pueda verificar si el módulo identificado por el sistema es el correcto.
  \item \textbf{[PYPH-RS-015]:} El sistema SLAM debe permitir mediante un mensaje en el canal de
  comunicación asociado a la IMU, solicitar un servicio de calibración de cada uno de sus
  sensores.
  \item \textbf{[PYPH-RS-016]:} El sistema SLAM debe permitir mediante un mensaje en el canal de
  comunicación asociado a la IMU, solicitar lecturas individuales de cada uno de los sensores
  de la IMU.
  \item \textbf{[PYPH-RS-017]:} El sistema SLAM debe permitir mediante un mensaje en el canal de
  comunicación asociado a la IMU, solicitar iniciar (y detener) un stream de lecturas en tiempo
  real de cada uno de los sensores de la IMU.
\end{itemize}

Referidos a la cámara:

\begin{itemize}
  \item \textbf{[PYPH-RS-018]:} El sistema SLAM debe permitir mediante un mensaje en el canal de
  comunicación asociado a la cámara, configurar la frecuencia de muestreo de esta.
  \item \textbf{[PYPH-RS-019]:} El sistema SLAM debe permitir mediante un mensaje en el canal de
  comunicación asociado a la cámara, reiniciar el driver de la cámara.
  \item \textbf{[PYPH-RS-020]:} El sistema SLAM debe permitir mediante un mensaje en el canal de
  comunicación asociado a la cámara, realizar un chequeo de sanidad para comprobar que esta se
  encuentra disponible. Ante tal mensaje el sistema responderá con otro mensaje en el mismo
  canal de comunicación, con un identificador de error si no se encuentra disponible, o caso
  contrario con información del módulo de la cámara (por ejemplo, ID y configuración actual) 
  para que el usuario pueda verificar si el módulo identificado por el sistema es el correcto.
  \item \textbf{[PYPH-RS-021]:} El sistema SLAM debe permitir mediante un mensaje en el canal de
  comunicación asociado a la cámara, solicitar lecturas individuales, es decir, imágenes.
  \item \textbf{[PYPH-RS-022]:} El sistema SLAM debe permitir mediante un mensaje en el canal de
  comunicación asociado a la cámara, solicitar iniciar (y detener) un stream de lecturas en
  tiempo real de la cámara, es decir, video.
\end{itemize}

Referidos al módulo de SLAM:

\begin{itemize}
  \item \textbf{[PYPH-RS-023]:} El sistema SLAM debe permitir mediante un mensaje en el canal de
  comunicación asociado a la configuración del módulo, configurar los distintos parámetros del
  algoritmo ``\textit{on-the-fly}''.
  \item \textbf{[PYPH-RS-024]:} El sistema SLAM debe permitir mediante un mensaje en el canal de
  comunicación asociado a la posición y orientación de la plataforma móvil, solicitar lecturas individuales del estado actual de posición y orientación de la plataforma móvil.
  \item \textbf{[PYPH-RS-025]:} El sistema SLAM debe permitir mediante un mensaje en el canal de
  comunicación asociado a la posición y orientación de la platafirma móvil, solicitar iniciar
  (y detener) un stream de lecturas en tiempo real de la posición y orientación de la plataforma
  móvil.
\end{itemize}

\subsection{Requisitos de rendimiento}
\label{sec:org94bc543}

N/A.

% Se detallarán los requisitos relacionados con la carga que se espera
% tenga que soportar el sistema. Por ejemplo, el número de terminales,
% el número esperado de usuarios simultaneamente conectados, número de
% transacciones por segundo que deberá soportar el sistema, etc.
%   También, si es necesario, se especificarpán los requisitos de
% datos, es decir, aquellos requisitos que afecten a la información
% que se guardará en la base de datos. Por ejemplo, la frecuencia de
% uso, las capacidades de acceso y la cantidad de registros que se
% espera almacenar (decenas, cientos, miles o millones).

\subsection{Restricciones de diseño}
\label{sec:org49fe900}


% Todo aquello que restrinja las decisiones relativas al diseño de la
% aplicación: Restricciones de otros estándares, limitaciones del
% hardware, etc.

\begin{itemize}
  \item \textbf{[PYPH-RS-026]:} La arquitectura de microprocesador que debe utilizar el sistema
  es ARM Cortex-M4.
  \item \textbf{[PYPH-RS-027]:} La primera versión del producto no debe considerar el uso de
  componentes como FPGA para la optimización del algoritmo utilizado.
  \item \textbf{[PYPH-RS-028]:} El sistema debe utilizar la librería micro-ROS para la gestión
  de la comunicación con el sistema usuario.
\end{itemize}

\subsection{Atributos del sistema}
\label{sec:orgd0babc0}

% Se detallarán los atributos de calidad (las "ilities") del
% sistema. Fiablidad, manteniblidad, portabilidad, y muy importante,
% la seguridad. Deberá especificarse qué tipos de usuarios están
% autorizados, o no, a realizar ciertas tareas, y cómo se
% implementarán los mecanismos de seguridad (por ejemplo, por medio de
% un \emph{login} y una \emph{password}).

\begin{enumerate}
  \item Mantenibilidad:
  \begin{itemize}
    \item \textbf{[PYPH-RS-029]:} Todo el software desarrollado para el sistema SLAM debe
    tener documentación automática con la herramienta Doxygen. Dicha documentación debe estar
    disponible públicamente en la web del proyecto.
    \item \textbf{[PYPH-RS-030]:} El desarrollo del software debe realizarse haciendo uso del
    sistema de control de versiones Git.
  \end{itemize}
  \item Modularidad:
  \begin{itemize}
    \item \textbf{[PYPH-RS-031]:} El sistema SLAM debe definir interfaces claras en la
    implementación del algoritmo, para permitir intercambiar el mismo con otro algoritmo
    que tenga mismo propósito pero diferentes características.
  \end{itemize}
  \item Portabilidad:
  \begin{itemize}
    \item \textbf{[PYPH-RS-032]:} El sistema SLAM debe definir archivos de port con el
    software utilizado que es específico para la arquitectura ARM Cortex-M. Esos archivos de
    port deben tener una interfaz de abstracción que permita la portabilidad del algoritmo
    a otras arquitecturas.
  \end{itemize}
\end{enumerate}



\subsection{Otros requisitos}
\label{sec:org31d2978}

% Cualquier otro requisito que no encaje en otra sección.

N/A.

\newpage


\section{Apéndices}
\label{sec:org75cea03}

% Puede contener todo tipo de información relevante para la ERS pero
% que, propiamente, no forme parte de la ERS. Por ejemplo:

% \begin{enumerate}
% \item Formatos de entrada/salida de datos, por pantalla o en listados.

% \item Resultados de análisis de costes.

% \item Restricciones acerca del lenguaje de programación.
% \end{enumerate}

\end{document}
